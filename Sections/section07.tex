\section{Непрерывные дроби}
    \begin{example}
        Заметим, что  $43=19\cdot 2+5$, $19=5\cdot 3+4$. Рассмотрим дробь:
        \begin{multline*}
            \cfrac{a}{b}=\cfrac{19}{43}=\cfrac{1}{\cfrac{43}{19}}=\cfrac{1}{\cfrac{2\cdot 19+5}{19}}=\cfrac{1}{2+\cfrac{5}{19}}=\cfrac{1}{2+\cfrac{1}{\cfrac{19}{5}}}=\cfrac{1}{2+\cfrac{1}{\cfrac{5\cdot 3+4}{5}}}=\\
            =\cfrac{1}{2+\cfrac{1}{3+\cfrac{1}{\cfrac{5}{4}}}}=\cfrac{1}{2+\cfrac{1}{3+\cfrac{1}{1+\cfrac{1}{4}}}}
        \end{multline*}
    \end{example}
    \begin{definition}
        Непрерывной (цепной) дробью будем называть выражение вида:
        \[[q_0; q_1, q_2, \dots q_n] = q_0=\cfrac{1}{q_1+\cfrac{1}{q_2+\cfrac{1}{\ddots+\cfrac{1}{q_n}}}}\ \ \ (\ast)\]
    \end{definition}
    \begin{theorem} \label{th7.1}
        Пусть $a$ - целое, $b$ - натуральное и пусть $(a,b)=1$. Пусть, кроме того, $q_0, q_1,\dots, q_n$ - все неполные частные, возникающие при отыскании $(a,b)$ с помощью алгоритма Евклида. Тогда число $\alpha=\frac{a}{b}$ разлагается в непрерывную дробь вида ($\ast$).
    \end{theorem} 
    \begin{proof}
        Доказательство следует из цепочки равненств:\\
        \[a=bq_0+r_1,\]
        \[b=r_1q_1+r_2,\]
        \[r_1=r_2q_2+r_3,\]
        \[\vdots\]
        \[r_{n-2}=r_{n-1}q_{n-1}+r_n\]
        Получаем:
        \begin{equation*}
            \cfrac{a}{b}=q_0+\cfrac{r_1}{b}=q_0+\cfrac{1}{\cfrac{b}{r_1}}=q_0+\cfrac{1}{q_1+\cfrac{r_2}{r_1}}=q_0+\cfrac{1}{q_1+\cfrac{1}{\cfrac{r_1}{r_2}}}=q_0+\cfrac{1}{q_1+\cfrac{1}{q_2+\cfrac{r_3}{r_2}}}=\dots
        \end{equation*}
        Собирая полученые равенства вместе приходим к $(\ast)$.
    \end{proof}
    \begin{example}\tab \\
        $a=37,\ b=8,\\
        37=8\cdot 4+5,\\
        8=5\cdot 1+3,\\
        5=3\cdot 1+2, \tab[1cm] \Rightarrow \alpha=\frac{37}{8}=[4;1,1,1,2].\\
        3=2\cdot 1+1,\\
        2=1\cdot 2$.
    \end{example}
    \begin{definition}
        Величины $q_0, q_1, \dots, q_n$ в разложении числа $\alpha=\frac{a}{b}$ из теоремы \ref{th7.1} называется неполным частным $b$ в разложении $\alpha$ в непрерывную дробь.
    \end{definition} 
        Дроби \[\delta_0=q_0\]
        \[\delta_1=q_0+\cfrac{1}{q_1}\] 
        \[\delta_2=q_0+\cfrac{1}{q_1+\cfrac{1}{q_2}}\]
        \[\vdots\]
    называются подходящими дробями.
    \begin{example}
    \[q_0=4,\ q_1=1,\ q_2=1,\ q_3=1, q_4=2.\]
    Тогда
    \[\delta_0=4\]
    \[\delta_1=4+\cfrac{1}{1}=5\]
    \[\delta_2=4+\cfrac{1}{1+\cfrac{1}{1}}=4+\cfrac{1}{2}=\cfrac{9}{2}\]
    \[\delta_3=4+\cfrac{1}{1+\cfrac{1}{1+\cfrac{1}{1}}}=4+\cfrac{2}{3}=\cfrac{14}{3}\]
    \end{example}
    \subsection*{Разложение иррационального числа в цепную дробь}
    
    Пусть $\alpha$ - число, не являющееся рациональным (такие числа будем называть иррациональными). Тогда для $\alpha$ тоже можно построить разложение в непрерывную дробь. Это разложение будет бесконечным (в отличии от рационального $\alpha=\frac{a}{b}$), поэтому такое построение требует определенной аккуратности и проводится в несколько шагов. На первом шаге строятся подходящие дроби, отвечающие числу $\alpha$, затем исследуются их свойства. В итоге доказывается сходимость последовательности подходящих дробей к числу $\alpha$, что и завершает построение.\\
    Этап первый:\\
    Определим целое $q_0$ так, чтобы выполнялись неравенства: \[q_0<\alpha<q_0+1\] и положим $\alpha_0=\alpha$, так что \[q_0<\alpha_0<q_0+1\] но тогда $\alpha_0=q_0+\beta_0$, где $0<\beta_0<1$ и, следовательно, \[\alpha_1=\frac{1}{\beta_0}>1\] и \[\alpha_0=q_0+\frac{1}{\alpha_1}\] Число $\alpha_1$, очевидно, иррационально, определим по нему целое число $q_1$ так, чтобы выполнялись неравенства: \[q_1<\alpha_1<q_1+1\] но $\alpha_1>1$, так что $q_1\geq 1$, т.е. $q_1$ - натуральное. Далее \[\alpha_1=q_1+\beta_1\] где $0<\beta_1<1$ и следовательно \[\alpha_2=\frac{1}{\beta_1}>1,\ \alpha_1=q_1+\frac{1}{\alpha_2},\ \alpha_0=q_0+\cfrac{1}{q_1+\cfrac{1}{\alpha_2}}\] Число $\alpha_2$ также иррационально. Повторяя это процесс, получим бесконечные последовательности иррациональных чисел $\alpha_1, \alpha_2,\dots, \alpha_{\xi},\dots$ (причем $\alpha_{\xi}>1$ для всех $\xi$) и натуральных чисел $q_1,\dots, q_{\xi}$ таких, что \[q_{\xi}=[\alpha_{\xi}]\ \text{и} \ \alpha_{\xi}=q_{\xi}+\frac{1}{\alpha_{\xi}+1}\] 
    Величины $q_0,q_1,q_2,\dots$ станем называть неполными частными разложения $\alpha$ в непрерывную дробь.
    Несложно видеть, что при любом $\xi$ справедливо равенство \[\alpha=q_0+\cfrac{1}{q_1+\cfrac{1}{q_2+\cfrac{1}{\ddots+\cfrac{1}{q_{\xi-1}+\cfrac{1}{\alpha_{\xi}}}}}}\]
    Определим по этим числам последовательность подходящих дробей $\delta_{\xi},\\
    \xi=0,1,\dots$ равенствами
    \[\delta_0=q_0,\ \ \delta_1=q_0+\cfrac{1}{q_1},\ \ \delta_2=q_0+\cfrac{1}{q_1+\cfrac{1}{q_2}},\ \ \dots\]
    \[\delta_{\xi}=q_0+\cfrac{1}{q_1+\cfrac{1}{q_2+\cfrac{1}{\ddots+\cfrac{1}{q_{\xi-1}+\cfrac{1}{\alpha_{\xi}}}}}}\]
    Выпишем первые три такие дроби:
    \[\delta_0=q_0,\ \ \delta_1=\cfrac{q_0 q_1+1}{q_1},\ \ \delta_2=\cfrac{q_0 q_1 q_2+q_0+q_2}{q_1 q_2+1}\]
    обозначим их еще так:
    \[\delta_0=\frac{P_0}{Q_0},\ \text{где}\ \ P_0=q_0,\ \ Q_0=1\]
    \[\delta_1=\frac{P_1}{Q_1},\ \text{где}\ \ P_1=q_0 q_1+1,\ \ Q_1=q_1\]
    \[\delta_2=\frac{P_2}{Q_2},\ \text{где}\ \ P_2=q_0 q_1 q_2+1,\ \ Q_2=q_1 q_2+1\]
    посмотрим как эти величины связаны между собой:
    \[P_1=q_1 P_0+1,\ \ Q_1=q_1 Q_0+0\]
    \[P_2=q_2 (q_0q_1+1)+q_0=q_2 P_1+P_0,\ \ Q_2=q_2 Q_1+Q_0\]
    Введем (формально) величины $P_{(-1)}=1, Q_{(-1)}=0$ (к подходящим дробям они не имеют отношения: выражение $\frac{P_{(-1)}}{Q_{(-1)}}=\frac{1}{0}$ не определено)\\
    Тогда равенства для $P_1, Q_1, P_2, Q_2$ запишутся единообразно:
    \[P_{\xi}=q_{\xi}P_{\xi-1}+P_{\xi-2},\ \ Q_{\xi}=q_{\xi}Q_{\xi-1}+Q_{\xi-2}\ \ \ (\xi=1,2)\]
    Оказывается, что эти соотношения верны для всех $\xi\geq 3$. Чтобы аккуратно доказать их, поступим следующим образом.
    \\Этап второй:\\
    Пусть даны переменные $x_0, x_1, x_2,\dots x_{\xi}, \dots$ произвольной природы (не обязательно целые числа). Рассмотрим величины $P_{\xi}$ и $Q_{\xi}$, определенные рекуррентными соотношениями
    \begin{equation} \label{equation3}
        \begin{cases}
            P_{\xi}=x_{\xi} P_{\xi-1}+P_{\xi-2}\\
            Q_{\xi}=x_{\xi} Q_{\xi-1}+Q_{\xi-2}
        \end{cases}
    \end{equation}
    ясно, что $P_{\xi}$ и $Q_{\xi}$ - некоторые многочлены от переменных $x_1,\dots x_{\xi},\dots$, например: $P_3-x_3P_2+P_1=x_3(x_0x_1x_2+x_0+x_2)+x_0x_1+1=x_0x_1x_2x_3+x_0x_1+x_0x_3+x_2x_3+1$, положим также $h_{\xi}=P_{\xi}Q_{\xi-1}-P_{\xi-1}Q_{\xi}$.
    \begin{lemma}\label{lemma7.1}
        При любом $\xi\geq 0$ справедливо равенство: $h_{\xi}=(-1)^{\xi-1}$
    \end{lemma} 
    \begin{proof}
        Индукция по $\xi$. В случае $\xi=0$ имеем: \[h_0=P_0Q_{(-1)}-P_{(-1)}Q_0=-P_{(-1)}Q_0=-1=(-1)^{0-1}\]
        Пусть соотношение доказано для всех $\xi\leq m$. Тогда 
        \begin{multline*}
            h_{m+1}=P_{m+1}Q_m-P_m Q_{m+1}=(x_{m+1}P_m+P_{m-1})Q_m-P_m(x_{m+1}Q_m+Q_{m-1})=\\=x_{m+1}(P_m Q_m-P_m Q_m)+P_{m-1}Q_m-P_m Q_{m-1}=-h_m=-(-1)^{m-1}=(-1)^m
        \end{multline*}
    \end{proof} 
    \begin{lemma}\label{lemma7.2}
        Если $x_0=q_0,\ x_1=q_1,\ x_{\xi}=q_{\xi}$ - целые числа, а величины $P_{\xi}$ и $Q_{\xi}$ определены в (3) то справедливы равентсва \[(P_{\xi},Q_{\xi})=(P_{\xi},P_{\xi-1})=(Q_{\xi},Q_{\xi-1})=1\]
    \end{lemma} 
    \begin{proof}
        Сразу следует из леммы \ref{lemma7.1}.
    \end{proof} 
    \begin{lemma}\label{lemma7.3}
        Пусть $x_1,\dots x_{\xi},\dots$ - произвольные переменные, и пусть выражения $\Delta_0,\dots \Delta_{\xi}\dots$ зависящие от $x_1,\dots x_{\xi},\dots$ определяются следующим образом: $\Delta_0=x_0$, а при $\xi\geq 1$ выражение для $\Delta_{\xi}$ получим, заменив в выражении для $\Delta_{\xi-1}\ \ x_{\xi-1}$ на $x_{\xi-1}+\frac{1}{x_{\xi}}$, так что, например,
        \[\Delta_1=x_0+\cfrac{1}{x_1},\ \ \Delta_2=x_0+\frac{1}{x_1+\cfrac{1}{x_2}},\ \ \Delta_3=x_0+\frac{1}{x_1+\cfrac{1}{x_2+\cfrac{1}{x_3}}}\]
        тогда при любом $\xi\geq 0$ справедливо равенство $\Delta_{\xi}=\frac{P_{\xi}}{Q_{\xi}}$, где P и Q определены соотношениями \eqref{equation3}
    \end{lemma} 
    \begin{proof}
        Индукция по $\xi$. В случае $\xi=0,1$ эти соотношения фактически были проверены ранее. Пусть они верны для всех $\xi\leq m$. Тогда
        \[\Delta_{\xi}=\frac{P_m}{Q_m}=\frac{x_m P_{m-1}+P_{m-2}}{x_m Q_{m-1}+Q_{m-2}}\]
        по определению, $\Delta_{m+1}$ получим из $\Delta_m$ заменой $x_m$ на $x_m+\frac{1}{x_{m+1}}$ переменная $x_m$, очевидно, не входит в выражения для $P_{m-1}, p_{m-2}, Q_{m-1}, Q_{m-2}$. Следовательно
        \begin{multline*}
        \Delta_{m+1}=\cfrac{(x_m+\cfrac{1}{x_{m+1}})P_{m-1}+P_{m-2}}{(x_m+\cfrac{1}{x_{m+1}})Q_{m-1}+Q_{m-2}}=\cfrac{(x_{m+1}x_m+1)P_{m-1}+x_{m-1}P_{m-2}}{(x_{m+1}x_m+1)Q_{m-1}+x_{m-1}Q_{m-2}}=\\=\cfrac{x_{m+1}(x_m P_{m-1}+P_{m-2})+P_{m-1}}{x_{m+1}(x_m Q_{m-1}+Q_{m-2})+Q_{m-1}}=\cfrac{x_{m+1}P_m+P_{m-1}}{x_{m+1}Q_m+Q_{m-1}}
        \end{multline*}
        но числитель и знаменатель последней дроби совпадают в силу \eqref{equation3} с $P_{m+1}$ и $Q_{m+1}$
    \end{proof} 
    \begin{theorem} \label{th7.2}
        Пусть $\alpha$ - произвольное вещественное число, и пусть $q_0, q_1, \dots$ - конечная или бесконечная последовательность неполных частных разложения $\alpha$ в непрерывную дробь. Тогда подходящие дроби $\delta_{\xi}, \ \xi=0,1,\dots$, отвечающие такому разложению, вычисляются по формулам
        \begin{equation} \label{equation4}
            \delta_{\xi}=\frac{P_{\xi}}{Q_{\xi}}
        \end{equation}
        где величины $P_{\xi}$ и $Q_{\xi}$ определяются следующими рекуррентными соотношениями: $P_{\xi}=q_{\xi}P_{\xi-1}+P_{\xi-2}, \ Q_{\xi}=q_{\xi}Q_{\xi-1}+Q_{\xi-2}$ с начальными условиями $P_{(-1)}=1,\ Q_{(-1)}=0,\ P_0=q_0,\ Q_0=1$. Все дроби \eqref{equation4} при этом несократимы.
    \end{theorem} 
    \begin{proof}
        Равенство \eqref{equation4} - есть прямое следствие леммы \ref{lemma7.1}
    \end{proof} 
    Если $\alpha \not\in \Q \Rightarrow q_0, q_1, q_2, \dots\ , \delta_{\xi}=\frac{P_{\xi}}{Q_{\xi}}$. Осталось непонятным, какое отношение имеют дроби $\delta_{\xi}$ к числу $\alpha$.\\
    Этап третий:
    \begin{lemma} \label{lemma7.4}
        При любом $\xi\geq 1$ верны неравенства: $\delta_{2\xi}>\delta_{2\xi-2}\ (\delta_{2\xi+1}<\delta_{2\xi-1})$, то есть подходящие дроби с четными (нечетными) номерами образуют монотонно возрастающую (убывающую последовательность).
    \end{lemma} 
    \begin{proof}
        \begin{multline*}
        \delta_{k}-\delta_{k-1}=\frac{P_k}{Q_k}-\frac{P_{k-1}}{Q_{k-1}}=\frac{P_k Q_{k-1}-P_{k-1}Q_k}{Q_k Q_{k-1}}= \text{(по лемме \ref{lemma7.1})}\\ =\frac{h_k}{Q_k Q_{k-1}}=\frac{(-1)^{k-1}}{Q_k Q_{k-1}}
        \end{multline*}
        тогда
        \begin{multline*}
            \delta_{2\xi}-\delta_{2\xi-2}=(\delta_{2\xi}-\delta_{2\xi-1})+(\delta_{2\xi-1}-\delta_{2\xi-2})=\frac{(-1)^{2\xi-1}}{Q_{2\xi}Q_{2\xi-1}} + \frac{(-1)^{2\xi-2}}{Q_{2\xi-1}Q_{2\xi-2}}=\\
            =\frac{1}{Q_{2\xi-1}}(\frac{1}{Q_{2\xi-2}}-\frac{1}{Q_{2\xi}})=\frac{Q_{2\xi}-Q_{2\xi-2}}{Q_{2\xi} Q_{2\xi-1} Q_{2\xi-2}}= (Q_{2\xi}=q_{2\xi} Q_{2\xi-1}+Q_{2\xi-2})\\
            =\frac{q_{2\xi} Q_{2\xi-1}}{Q_{2\xi} Q_{2\xi-1} Q_{2\xi-2}}=\frac{q_{2\xi}}{Q_{2\xi}Q_{2\xi-2}}>0
        \end{multline*}
        Неравенство $\delta_{2\xi+1}-\delta_{2\xi-1}$ доказывается аналогично.
    \end{proof} 
    \begin{lemma} \label{lemma7.5}
        В условиях леммы \ref{lemma7.4} справедливы неравенства: $\delta_{\xi}<\alpha,\ \xi$ - четное и $\delta_{\xi}>\alpha,\ \xi$ - нечетное.
    \end{lemma} 
    \begin{proof}
        Рассмотрим выражения
        \[\alpha=q_0+\cfrac{1}{q_1+\cfrac{1}{q_2+\cfrac{1}{\ddots+\cfrac{1}{q_{\xi}+\cfrac{1}{\alpha_{\xi+1}}}}}}\]
        \[\delta_{\xi+1}=q_0+\cfrac{1}{q_1+\cfrac{1}{q_2+\cfrac{1}{\ddots+\cfrac{1}{q_{\xi}+\cfrac{1}{q_{\xi+1}}}}}},\ \ \delta_{\xi}=q_0+\cfrac{1}{q_1+\cfrac{1}{q_2+\cfrac{1}{\ddots+\cfrac{1}{q_{\xi}}}}}\]
        выражения для $\alpha$ и $\delta_{\xi+1}$ получаются из выражения для $\delta_{\xi}$ формальной заменой $q_{\xi}$ на $q_{\xi}+\frac{1}{\alpha_{\xi+1}}$ и на $q_{\xi}+\frac{1}{q_{\xi+1}}$ соответственно.
        \[\delta_{\xi}=\frac{P_{\xi}}{Q_{\xi}}=\frac{q_{\xi}P_{\xi-1}+P_{\xi-2}}{q_{\xi}Q_{\xi-1}+Q_{\xi-2}} \Rightarrow \alpha=\cfrac{(q_{\xi}+\cfrac{1}{\alpha_{\xi+1}})P_{\xi-1}+P_{\xi-2}}{(q_{\xi}+\cfrac{1}{\alpha_{\xi+1}})Q_{\xi-1}+Q_{\xi-2}}=\frac{A_{\xi}}{B_{\xi}}\]
        \[\delta_{\xi+1}=\cfrac{(q_{\xi}+\cfrac{1}{q_{\xi+1}})P_{\xi-1}+P_{\xi-2}}{(q_{\xi}+\cfrac{1}{q_{\xi+1}})Q_{\xi-1}+Q_{\xi-2}}=\frac{P_{\xi+1}}{Q_{\xi+1}}\]
        вычислим:
        \[\alpha-\delta_{\xi+1}=\frac{A_{\xi}}{B_{\xi}}-\frac{P_{\xi+1}}{Q_{\xi+1}}=\frac{A_{\xi}Q_{\xi+1}-B_{\xi}P_{\xi+1}}{B_{\xi}Q_{\xi+1}}\]
        числитель:
        \begin{multline*}
            \ast = (q_{\xi}+\frac{1}{q_{\xi+1}})(q_{\xi}+\frac{1}{\alpha_{\xi+1}})(P_{\xi-1}Q_{\xi-1}-P_{\xi-1}Q_{\xi-1})+(P_{\xi-2}Q_{\xi-2}-P_{\xi-2}Q_{\xi-2})+\\
            +(q_{\xi}+\frac{1}{q_{\xi+1}})+(P_{\xi-2}Q_{\xi-1}-P_{\xi-1}Q_{\xi-2})+(q_{\xi}+\frac{1}{\alpha_{\xi+1}})+(P_{\xi-1}Q_{\xi-2}-P_{\xi-2}Q_{\xi-1})
        \end{multline*}
        итак, числитель разности $\alpha-\delta_{\xi+1}$ равен
        \begin{multline*}
            (P_{\xi-1}Q_{\xi-2}-P_{\xi-2}Q_{\xi-1})(q_{\xi}+\frac{1}{\alpha_{\xi+1}}-q_{\xi}-\frac{1}{q_{\xi+1}})=h_{\xi-1}(\frac{1}{\alpha_{\xi+1}}-\frac{1}{q_{\xi+1}})=\\
            =(-1)^{\xi-2}\frac{q_{\xi+1}-\alpha_{\xi+1}}{\alpha_{\xi+1}q_{\xi+1}}=\frac{(-1)^{\xi+1}\alpha_{\xi+1}}{\alpha_{\xi+1}q_{\xi+1}}
        \end{multline*}
        $\Rightarrow$ знак разности $\alpha-\delta_r$ совпадает с $(-1)^r$
    \end{proof} 
    \begin{theorem}\label{th7.3}
        Последовательность подходящих дробей, отвечающих разложению иррационального числа $\alpha$ в непрерывную дробь, сходится к числу $\alpha$.
    \end{theorem} 
    \begin{proof}
        По лемме \ref{lemma7.4} последовательность $\delta_{2\xi},\ \xi=0,1,2,\dots$ монотонно возрастает. По лемме \ref{lemma7.5} она ограничена сверху числом $\alpha$. Аналогично, последовательность $\delta_{2\xi+1}$ монотонно убывает и ограничена снизу числом $\alpha$. По известной теореме из математического анализа, эти последовательности имеют пределы. По лемме \ref{lemma7.5}\ : $\delta_{2\xi}<\alpha<\delta_{2\xi+1}$ при любом $\xi \geq 0$. Значит,
        \[0<\alpha-\delta_{2\xi}<\delta_{2\xi+1}-\delta_{2\xi}=\frac{1}{Q_{2\xi}Q_{2\xi+1}},\ \ 0<\delta_{2\xi+1}-\alpha<\delta_{2\xi+1}-\delta_{w\xi}=\frac{1}{Q_{2\xi}Q_{2\xi+1}}\]
    \end{proof} 
    \begin{example}
        Пусть $\tau = \frac{1+\sqrt{5}}{2}>1,6$. Докажем, что $Q_{\xi}\geq \tau^{\xi-1}$\\
        Индукция по $\xi$. База: $Q_1=1=\tau^{1-1}$. Пусть доказано для всех $\xi: q\leq \xi \leq m$.\\
        $\tau^2=\tau+1 \Rightarrow \tau^{k+2}=\tau^{k+1}+\tau^k$ для всех $k\geq 0$. Тогда
        \[Q_{m+1}=q_{m+1}Q_m+Q_{m-1}\geq Q_m+Q_{m-1}\geq \tau^{m-1}+\tau^{m-2}=\tau^m\]         
        \begin{multline*}
            Q_{2\xi}Q_{2\xi-1}\geq \tau^{2\xi-1}\tau^{2\xi-2}=\tau^{4\xi-3}=\frac{\tau^{4\xi}}{\tau^3},\ 0<\alpha-\delta_{2\xi},\ \delta_{2\xi+1}-\alpha<\frac{\tau^{3}}{\tau^{4\xi}}=\\
            =\frac{2+\sqrt{5}}{(\frac{7+3\sqrt{5}}{2})^{\xi}}<\frac{5}{6^{\xi}}\to 0\ \ \ (\xi\to +\infty)
        \end{multline*}
    \end{example}  
    \begin{comm}
        Последовательность неполных частных $q_0, q_1, q_2, \dots$ (бесконечная) периодична (начиная с некоторого номера) $\lra \alpha$ - квадратичная иррациональность, то есть $\alpha=\frac{A+B\sqrt{D}}{C},\ A,B,C,D\in \Z,\ D\geq 1$ - бесквадратное.
        Например:
        \[\sqrt{2}=[1;2,2,\dots]=1+\cfrac{1}{2+\cfrac{1}{2+\cfrac{1}{\ddots}}}\] 
    \end{comm} 
    \begin{comm}
        Лишь для немногих $\alpha\not\in \Q$, не являющихся квадратичной иррациональностью, известны разложения в цепную дробь.\\
    \end{comm}
    \begin{examples}
        Число $e=2,718281828459045\dots$
        \[\alpha=\cfrac{e-1}{e+1}=[0;2,6,10,14,18,22,\dots]\]
        \[\alpha=\cfrac{e^{\frac{2}{k}}-1}{e^{\frac{2}{k}}+1}=\th{(e^{\frac{1}{k}})}=[0;1k,3k,5k,7k,\dots]\ \ \ \ \text{(Л. Эйлер, 1737г.)}\]
        \[e=[2;1,2,1,1,4,1,1,6,1,1,8,\dots]\ \ \ \ \text{(Л. Эйлер, 1737г.)}\]
        \[\alpha=\sqrt[3]{2}=[1;3,1,5,1,1,4,1,1,8,\dots]\]
        \[\alpha=\sqrt[3]{6}=[1;1,4,2,7,3,508,1,5,5,\dots]\]
        \[\pi=[3;7,5,1,292,1,1,1,2,1,\dots]\]
    \end{examples}
    \begin{comm}
        Рассмотрим $N\geq 3$ и дроби $\frac{a}{N},\ 1\leq a\leq N-1,\ (a,N)=1$ (таких дробей $\phi(N)$ штук). Разложим каждую в непрерывную дробь:
        \[\frac{a}{N}=\cfrac{1}{q_1+\cfrac{1}{q_2+\cfrac{1}{\ddots+\cfrac{1}{q_n}}}}\]\
        где $n=n(a)$ - длина разложения.
        Вопрос: Каково среднее значение $n(a)$ при изменении $a$?
        \[\frac{1}{\phi(N)}\sum\limits_{a=1}^N n(a)\approx \frac{12}{\pi^2}(\ln{2})+A,\ \ \text{A - некоторая константа}\ \ \ \text{(Х. Хейльбрин, 1969г.)}\]
    \end{comm}