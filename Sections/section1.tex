\section*{Введение}
        Следующие понятия считаются интуитивно ясными:
    \begin{enumerate}
        \item Понятие натурального ряда $\mathbb{N} = \{1, 2, 3, 4, 5, \dots\}$.
        \item У каждого натурального числа $n$ существует единственное натуральное число $m=n+1$ следующее за ним. 
        \item Понятие отрицательных чисел и нуля.
        \item Понятие суммы, разности и произведения двух целых чисел.
    \end{enumerate}
    \begin{axiom}
        Если $M \subset \mathbb{N}$ обладает следующими свойствами: $(1 \in M)$ и $(\forall n\in M$ выполнено $n+1 \in M)$, то $M = \mathbb{N}$.
    \end{axiom}
    \begin{cons}
        Всякое непустое подмножество натурального ряда содержит минимальный элемент.
    \end{cons}
    \begin{cons}
        Всякое непустое конечное подмножество натурального ряда содержит максимальный элемент.
    \end{cons}
    \begin{cons} (Принцип математической индукции)\newline
        Если известно, что некоторое утверждение о натуральных числах выполнено для натурального числа $a$, а также из предположения о том, что утверждение верно при некотором $n$ следует справедливость этого утверждения и для числа $n+1$, то это утверждение верно для всех натуральных чисел, больше или равных $a$.
    \end{cons}

\section{Делимость целых чисел}
    \begin{definition}
        Пусть $a,b\in \mathbb{N}, b\ne 0$. Говорят что $a$ делится на $b$, если существует $c\in \mathbb{Z}$, такое, что $a=bc$.
    \end{definition}
    \begin{comm}
        $a$ называется делимым, а $b$ называется делителем числа $a$. Запись $b \mid a$ означает, что $b$ делит $a$. Если $b$ не делит $a$, то пишут $b\ndiv a$.
    \end{comm}
    \begin{lemma} \label{lemma1.1}
        Пусть $a,b,c \in \mathbb{Z}$, тогда:
        \begin{enumerate}
            \item $1 \mid a$.
            \item $a\ne 0 \Rightarrow a\mid a$.
            \item $a\mid b \Rightarrow a\mid bc$.
            \item $a\mid b$ и $b\mid c \Rightarrow a\mid c$.
            \item $a\mid b$ и $a\mid c \Rightarrow a\mid (b+c)$.
            \item $a\mid b$ и $b\ne 0 \Rightarrow |a|\leq|b|$.
        \end{enumerate}
    \end{lemma}
    \begin{theorem} \label{th1.1}
        Если $a\in \Z, b\in \N$, то существует единственная пара целых чисел $q$ и $r$, таких, что $a=bq+r$, где $0\leq r\leq b-1$. 
    \end{theorem}
    \begin{proof}
        Докажем существование: Если $a$ делится на $b$, то $a=bc$. В таком случае возьмем $q=c, r=0$. Теперь пусть $a$ не делится на $b$. Рассмотрим непустое множество $M$ натуральных чисел, представимых в виде $a-kb, k\in \mathbb{Z}$, возьмем $k=-(|a|+1)$, тогда $a-kb=b(|a|+1)+a\geq b(|a|+1)-|a|\geq 1\cdot (|a|+1)-|a|=1 \Rightarrow a-kb$ - натуральное. Значит, у $M$ есть минимальный элемент $a-kb$. Возьмем $q=k, r=a-kb=a-bq > 1$. Осталось показать, что $0\leq r\leq b-1$. Предположим, что $r\geq b$. Если $r=b$, то $a=bq+b=b(q+1)$ получаем противоречие, так как $a$ не делится на $b$. Значит, $r=b+m, m\geq 1$. Получаем $1\leq m=r-b<r$, при этом $a=bq+r=bq+b+m=b(q+1)+m \Rightarrow m=a-b(q+1) \Rightarrow m\in M$ и $m<r$, получаем противоречие, так как $a$ не делится на $b$. Доказано, что $r<b \Rightarrow$ представление $a=bq+r$ - искомое. Докажем единственность: предположим, что для некоторых $a$ и $b$ имеются пары чисел с указанным свойством: $q,r$ и $q_1,r_1$, причем $0\leq r\leq r_1 \leq b-1$. Тогда $a=bq+r=bq_1+r_1\Rightarrow 0\leq b(q-q_1)=r_1-r$. Значит, $b$ делит разность $r_1-r$. Допустим, что $q\ne q_1$, тогда по пункту 6 леммы \ref{lemma1.1} получаем $b\leq r_1-r$ и в то же время $r_1-r\leq b-1<b$. Получаем противоречие, поэтому $q=q_1$, следовательно, и $r=r_1$.
    \end{proof}