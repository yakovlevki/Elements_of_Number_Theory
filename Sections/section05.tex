\section{Простые числа}
    \begin{definition}
        Натуральное число $n>1$ называется простым, если оно имеет ровно два делителя: $1$ и $n$. В противном случае это число называется составным.
    \end{definition}
    \begin{comm}
        Единица не причисляется ни к простым, ни к составным.
    \end{comm}
    \begin{lemma} \label{lemma5.1}
        Наименьший делитель натурального числа $n>1$, отличный от единицы - простое число.
    \end{lemma} 
    \begin{proof}
        Пусть $d\div n, 1 \textless d \leq n$, и $d$ - наименьший делитель с этими свойствами. Пусть $d$ - составное. Тогда $\exists\ k: k\div d$ и $1<k<d$. По лемме \ref{lemma1.1} $k\div n$, но $1<k<d$ - противоречие с тем, что $d$ - минимальный.
    \end{proof} 
    \begin{theorem} \label{th5.1}
        Множество простых чисел бесконечно.
    \end{theorem}
    \begin{proof}
        Пусть множество простых конечно: $p_1<p_2<\dots<p_n$ - все простые числа. Рассмотрим число $N = p_1p_2\dots p_n+1$. По лемме \ref{lemma5.1} наименьший делитель $p>1$ числа $N$ - простое число. Но $p$ отлично от $p_1\dots p_k, p$ делит $N$ нацело, а $N$ при делении на каждое из $p_1\dots p_n$ дает остаток $1$ - противоречие.
    \end{proof} 
    \begin{definition}
        Пусть $x>0$, через $\pi(x)$ обозначим количество простых чисел на отрезке $[0, x]$ ($\pi(x)$ - количество простых чисел не превосходящих $x$).
    $$\pi(x)=\sum\limits_{p\leq x}1$$
    \end{definition} 
    (Теорема \ref{th5.1}) $\lra \pi(x)$ - не ограничена сверху $\lra \pi(x)\to +\infty$ при $x\to +\infty$.\\
    Гипотеза Лежандра: $\pi(x)=\frac{x}{\ln x-C}$, где $C=1,08366$. Позднее Гаусс выдвинет более сложное и более точное предположение.
    Из доказательства теоремы Чебышева: $\forall \epsilon >0\ \exists\ x_0=x_0(\epsilon)$, т.ч. $\forall x\geq x_0$ выполнено неравенство:
    $$(A-\epsilon)\frac{x}{\ln x}<\pi(x)<(B+\epsilon)\frac{x}{\ln x}$$
    $$A=\ln (\frac{2^{\frac{1}{2}} \ 3^{\frac{1}{3}} \ 5^{\frac{1}{5}}}{30^{\frac{1}{30}}}), B=\frac{6}{5}A$$
    Асимптотический закон распределения простых чисел:
    $$\lim\limits_{x\to +\infty}(\frac{\pi(x)}{\frac{x}{\ln x}})=1 \lra A=B=1 \lra \pi(x)=(1+\om(1))\frac{x}{\ln x}$$
    \begin{lemma} \label{lemma5.2}
        Пусть $N$ - составное число, $p$ - наименьший простой делитель. Тогда $p\leq \sqrt{N}$.
    \end{lemma} 
    \begin{proof}
        $N$ - составное $\Rightarrow \exists\ a,b: 1<a<N,\ 1<b<N$ и $ab=N$. Значит $a\div N,\ b\div N, p$ - наименьший $\Rightarrow p\leq a,\ p\leq b \Rightarrow p^2\leq ab=N\\
        \Rightarrow p\leq \sqrt{N}$.
     \end{proof}
    \begin{eratosphene}
        Выписываем все числа от $2$ до $N$, первое число в таблице - простое, это 2. Вычеркиваем все числа кратные 2, кроме нее самой. Первое невычеркнутое число после 2 - это 3 - значит оно простое. Вычеркиваем все числа, кратные 3, кроме самой 3. Первое невычеркнутое число после 3 - простое и т.д. После того как найдено наибольшее простое $p$ не превосходящее $\sqrt{N}$ и вычеркнуты все числа кратные $p$, в таблице останутся лишь простые числа, не превосходящие $N$ и только они.
    \end{eratosphene}
    \begin{theorem} \label{th5.2}
        (Основная теорема арифметики)\\
        Каждое целое число, большее $1$, раскладывается в произведение простых чисел, притом единственным способом (с точностью до порядка сомножителей). 
    \end{theorem}
    \begin{proof}
        Существование:\\ Индукция по $n>1$. Числа $n=2, n=3$ - простые, для них это утверждение справедливо. Пусть $n>3$, и допустим, что справедливость утверждения проверена для всех $m<n$. Если $n$ - простое, то утверждение очевидно. Пусть $n$ - составное. По лемме \ref{lemma5.1} его наименьший делитель - простое число $\Rightarrow n=p_1k$, но $k=\frac{n}{p_1}\leq \frac{n}{2}<n$. По предположению индукции $k=p_2\dots p_r$, где $p_2,\dots,p_r$ - простые. $\Rightarrow n=p_1k=p_1p_2\dots p_r$ - искомое разложение.\\
        Единственность:\\
        Пусть $n=p_1\dots p_r=q_1\dots q_s,$ где $p_i,q_i$ - простые числа и $r\leq s$. Тогда\\ $p_1\dots p_r=q_1a_1$, где $a_1=q_2\dots q_s \Rightarrow p_1 \div q_1a_1$. Возможно два случая:\\
        1) $(p,q)>1 \Rightarrow p_1=q_1$.\\
        2) $(p,q)=1 \Rightarrow$ (теорема \ref{th2.3}) $p_1 \div a_1=q_2\dots q_s, a_1=q_2a_2, a_2=q_3\dots q_s,\\ p_1 \div q_2a_2\Rightarrow$ либо $p_1=q_2$, либо $p_1\div a_2$ и т.д. Но $a_1>a_2>\dots\geq 1 \Rightarrow$ на одном из шагов обязательно будет иметь место равенство $p_1=q_k, k\leq s$ (иначе оказалось бы, что $p_1 \div 1$, а это невозможно). Итак, $p_1$ совпадает с одним из чисел $q_1,\dots,q_s$. Будем считать, что $p_1=q_1 \Rightarrow p_2\dots p_r=q_2\dots q_s$ продолжаем рассуждение и получаем, что $p_2$ совпадает с одним из $q_2,\dots q_s$, пусть $p_2=q_2$ и т.д. Если $r<s$ после $r$ шагов получили бы противоречивое равенство: $1=q_{r+1}\dots q_s \\ \Rightarrow r=s$ и множества $\{p_1,\dots, p_r\}$ и $\{q_1, \dots , q_s\}$ совпадают.
    \end{proof}
    \begin{comm}
        $n>1, n=q_1\dots q_s \Rightarrow n$ можно записать в виде $n=p_1^{\alpha_1}\dots p_k^{\alpha_k}, \\ p_1<p_2<\dots<p_k$ - каноническое разложение $n$ на простые сомножители.
    \end{comm} 
    \begin{definition}
        Пусть $n=p_1^{\alpha_1}\dots p_k^{\alpha_k}, p$ - простое. Тогда
        \[\nu_p(n)=
        \begin{cases}
            0,\ \text{если}\ p\ndiv n,\\
            \alpha,\ \text{если} \ p=p_i. 
        \end{cases}\]
    \end{definition}    
    \begin{lemma} (Свойства $\nu_p(n)$) \label{lemma5.3}
        \begin{enumerate}
            \item Для любых целых чисел $a,b>1$ и любого простого $p$, справедливо равенство: $\nu_p(ab)=\nu_p(a)+\nu_p(b)$.
            \item Пусть $m,n>1$ - целые числа, тогда $m \div n \lra \nu_p(m)\leq \nu_p(n)$ для любого простого $p$.
        \end{enumerate}
    \end{lemma}
    \begin{proof} \tab
        \begin{enumerate}
            \item При перемножении степеней с одинаковыми основаниями, их показатели складываются.
            \item    
            \begin{itemize}
                \item[$(\Rightarrow)$] Пусть $n=km \Rightarrow \nu_p(n)=\nu_p(k)+\nu_p(m)\geq \nu_p(m)$.
                \item[$(\Leftarrow)$] Все разности $\nu_p(n)-\nu_p(m)$ - целые неотрицательные. Рассмотрим число:
                $$k = \prod\limits_pp^{\nu_p(n)-\nu_p(m)}$$ 
                Если $k=1$, то $\nu_p(n)=\nu_p(m)$ для всех $p$ и $m=n$. В силу основной теоремы арифметики, в этом случае $m \div n$. Пусть $k>1$, тогда в силу пункта 1:
                $$km=\prod\limits_pp^{\nu_p(n)-\nu_p(m)}\cdot \prod\limits_pp^{\nu_p(m)}=\prod\limits_pp^{\nu_p(n)}=n$$
                то есть $m \div n$. 
            \end{itemize}
        \end{enumerate}
    \end{proof}
    \begin{lemma}
        Для любых $a,b\in \N$ справедливы равенства:
        $$[a,b]=\prod\limits_pp^{\max(\nu_p(a),\nu_p(b))}$$
        $$(a,b)=\prod\limits_pp^{\min(\nu_p(a),\nu_p(b))}$$
    \end{lemma}
    \begin{proof}
        Обозначим $K=[a,b],$ $N=\prod\limits_pp^{\max(\nu_p(a),\nu_p(b))}$, поскольку\\ $\nu_p(a)\leq \nu_p(N),$ $\nu_p(b)\leq \nu_p(N)$, то $a$ и $b$ делят $N$ в силу леммы \ref{lemma5.3}. Значит $N$ - общее кратное чисел $a$ и $b$. С другой стороны, поскольку $a$ и $b$ делят $K$, то по лемме \ref{lemma5.3} имеем $\nu_p(a)\leq \nu_p(K),\nu_p(b)\leq \nu_p(K)$, так что $\nu_p(K)\geq \max(\nu_p(a), \nu_p(b))=\nu_p(N)$ для любого простого $p$. Значит, $N\div K$, но $N\leq K \Rightarrow N=K$. Вторая часть утверждения следует из первой, если воспользоваться равенством
        $$(a,b)=\frac{ab}{[a,b]}$$
        и тем, что $x+y=\max(x,y)+\min(x,y)$ $\forall x,y\in \R$.
    \end{proof}